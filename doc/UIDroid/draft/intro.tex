\section{Introduction} \label{sec: intro}
The exponential growth of mobile devices has raised security concerns.
Compared to malware, authentic apps still perform their proposed functionality.
However, they may also put users at risk by behaving in a user-unexpected way, such as stealthily sending user�s private information out for purposes such as analytics, advertising, cross-application profiling, and social computing \cite{}.
Static and dynamic program taint analysis of apps focus on identifying whether sensitive data leaves the user device.
They treat all sensitive transmission as illegal, so that suffer from high false positive rate.

In this paper, we propose a novel approach to detect unauthorized transmissions that are not required by the app's core business logic. 
We search program traces who generate sensitive transmissions within an app and keep track their effects on app user interfaces.
The program traces, along with their effects on user interfaces, are then fed to classifiers to train a model that helps to label further traces.  

\section{Motivating Example}
The intuition behind the method to differentiate between the legitimate and illegitimate transmissions:
\begin{itemize}
  \item \textbf{Exception}: If the sensitive resources (e.g location) could not be acquired, the legal transmissions would typically lead to alarms that visible to users, whereas the illegal transmissions would not notify users.
  \item \textbf{Layout}: Unnecessary feedbacks such as advertisements caused by illegitimate transmission are normally located at the corner of the screen. In contrary, the content that carries the core functions of the app occupies the central part of the window.
  \item \textbf{Text}: Useful visible content are generally surrounded by the descriptive text. Such phenomenon is rarely happen for the content generated by illegal transmissions.     
\end{itemize}

